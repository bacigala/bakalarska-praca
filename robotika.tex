\chapter{Robotika}

\label{kap:robotika}

V kapitole uvádzame základné pojmy z oblasti, prehľad vybraných historických udalostí vedúcich k vzniku dnešných robotov a krátke predstavenie ich súčasného využitia. Zameriavame sa i na výučbu robotiky u nás na Slovensku, predovšetkým na možnosti štúdia a dostupnosť potrebných zdrojov pre mladších záujemcov.

%Definície pojmov
\section{Definície pojmov}
\textbf{\textit{Robot}} je definovaný Medzinárodnou organizáciou pre normalizáciu (angl. International organization for standardization - ISO) ako \uv{spustiteľný mechanizmus programovateľný v dvoch alebo viacerých osiach pohybu, so stupňom autonomity, pohybujúci sa v prostredí s účelom vykonať plánované úkony}. Norma ďalej uvádza, že súčasťou robota je riadiaci systém s rozhraním umožňujúcim obsluhu a tiež definuje základné rozdelenie na priemyselné a obslužné roboty \cite{ISORobot}. Existuje viacero definícií, prienikom je požiadavka na programovateľnosť, autonomitu a schopnosť interakcie s prostredím. Niektoré kladú dôraz na podobnosť človeku, iné na viacúčelovosť \cite{RoboticsAndAutomationHandbook}. Interakcia s prostredím znamená komunikáciu medzi robotom a prostredím. Robot získava informácie z prostredia použitím receptorov, rôznych senzorov merajúcich veličiny ako teplota, vzdialenosť, tlak či rýchlosť. Komunikácia opačným smerom prebieha pomocou efektorov, ktorými robot vykonáva akcie, zväčša sú efektormi motory rôznych typov.

\textbf{\textit{Robotika}} je veda zaoberajúca sa robotmi.

\textbf{\textit{Stupeň voľnosti}} (angl. degree of freedom - DOF) označuje pohyblivosť. Plne pohyblivé teleso má šesť stupňov voľnosti, posun a rotáciu v každej z osí 3D priestoru. DOF možno ľahko ilustrovať kĺbmi ľudského tela, napríklad koleno má jeden stupeň voľnosti. V robotike sa zvykne udávať DOF pre robota ako súčet DOF jeho kĺbov.


% História
\section{História}
S veľkou pravdepodobnosťou sa môžeme domnievať, že prvý zrod robota, ako ho poznáme a definujeme dnes, bol len fikciou, nápadom či snom v mysli jedného z našich predkov \cite{RoboticsAndAutomationHandbook}. Výskyt prvého zariadenia (stvorenia) pripomínajúceho robota možno hľadať i nájsť v mytológií, ktorá priam prekvitá postavami záhadne sa preberajúcimi k životu, pripomínajúcimi ľudí plniacich najrôznejšie úlohy. Slovo \textit{robot} ako také sa objavuje až v roku 1920, v diele Karla Čapka --- Rossum's Universal Robots, údajne odvodené zo slova \uv{robotník} alebo \uv{robota}. Dielom autor upozorňuje na zneužitie robotiky, ale tiež ukazuje jej silu a potenciál. Robot sa krátko na to, v roku 1926, vyskytne i v kinematografii, konkrétne v sci-fi filme Fritza Langa s názvom Metropolis. Robotiku v literatúre neskôr reprezentoval známy Isaac Asimov, ktorý v svojich dielach zaviedol tri zákony robotiky podriaďujúce robotov človeku. Poukázal však aj na nedostatky a možné riziká spolužitia ľudstva a spoločnosti robotov.

Za dôveryhodnejších predchodcov dnešných robotov možno považovať rôzne mechanické vynálezy. Za jedno z prvých komplexnejších zariadení, ktoré sa môže uchádzať o toto označenie vzniklo v roku 270 p.n.l., keď v Grécku fyzik Ctesibus vytvoril prístroj na meranie času s vodným pohonom \cite{RoboticsAndAutomationHandbook}. Vynálezov vzniklo pochopiteľne veľa, presuňme sa ale do doby, keď sa do oblasti prvýkrát vniesla možnosť programovateľnosti, nakoniec, je to jedna z podmienok stanovených v definícii pojmu \uv{robot} \cite{ISORobot}. Zlatý čas pre pokrok a inovácie, najmä v mechanizácií, známy tiež ako priemyselná revolúcia, priniesol ďalší výrazný míľnik na začiatku devätnásteho storočia. Tentoraz v odvetví spracovania hodvábu, keď vo Francúzsku Joseph Jacquard prvýkrát použil dierne štítky na ovládanie vzoru produkovaného šijacím strojom. Neskôr princíp štítkov použil Charles Babbage pri konštrukcií automatickej kalkulačky a analytického stroja, pre ktorý je často označovaný ako otec počítačov. Pre zaujímavosť dodáme, že mimo šijacieho stroja patrili k prvým oblastiam aplikácie automatizácie napríklad výroba skrutiek alebo konštrukcia žeriavu. Robotike výrazne pomohol i preslávený elektrotechnický inžinier Nikola Tesla, najmä pre svoje prínosy v súvise s objavom striedavého prúdu.

Po druhej svetovej vojne prispeli k vývoju v oblasti robotiky najmä nové vynálezy ako servomotor ale i pokroky v rozvoji digitálnej technológie. V roku 1954 si George C. Devol jr. patentoval vynález umožňujúci cyklickú kontrolu zariadení, ktorú po strete s odborníkom Josephom Engelbergerom využili pri vývoji prvého priemyselného robota. Robot s názvom \uv{Unimate} vznikol v roku 1961, nahradil prácu robotníkov pri obsluhe stroja tvoriaceho odliatky a neskôr bol model použitý aj pri výrobe automobilov. Unimate sa na svojej pozícii osvedčil a začal tak éru automatizácie výrobných procesov pomocou robotiky. Oblasti aplikácie priemyselných robotov zahŕňali najmä zváračstvo ale známe boli aj roboty vykonávajúce maliarske práce.

V ďalšom období bolo v robotike snahou umožniť prístrojom vyššiu mieru interakcie s prostredím. Jeden z prvých pokusov o dosiahnutie tohto cieľa vznikol v Centre umelej inteligencie v Kalifornii. Centrum vyvinulo v rokoch 1966 až 1972 robota Shakey. Robot bol riadený počítačmi, kolesá mu umožnili pohyb v priestore, okolie vnímal videokamerou a niekoľkými senzormi. Okrem jednoduchého pohybu bol teda schopný aj základnej interakcie s prostredím a tým sa výrazne odlíšil od dovtedy známych zariadení. V tejto dobe sa o rozvoj vo veľkej miere pričinili i japonské spoločnosti, ktoré začali roboty nasadzovať do výrobných procesov.

Prvé roboty schopné komplexnejšej interakcie s prostredím narazili na komplikácie v súvislosti s potrebou náročnejších výpočtov, čo sa prejavilo na rýchlosti ich reakcií. Prvý systém pohybujúci sa na základe vyhodnotenia pozícií prekážok v okolí zo záberov z kamier sa pohyboval rýchlosťou jeden meter za 10 až 15 minút \cite{RoboticsAndAutomationHandbook}. Zrejmé prednosti a potenciál robotov však nezostali bez povšimnutia a s rozvojom sa pokračovalo. Vznikli rôzne inštitúcie zaoberajúce sa robotikou, pomohol i rozsiahly vývoj v oblasti informatiky. Zvyšovala sa presnosť a rýchlosť, na riadenie sa začali používať dnes celkom bežné mikropočítače.

V snahe vysporiadať sa so stúpajúcimi nárokmi na výpočty vznikli i nové oblasti, napríklad \uv{robotika založená na správaní} (angl. behavior--based robotics). Riadenie v \uv{klasickej} robotike je založené na vopred definovaných (komplexných) modeloch, každý príkaz znamená vykonanie preddefinovanej postupnosti akcií efektorov, existuje v nej teda akýsi \uv{centrálny modul riadenia} \cite{BehaviorBasedRobotics}. Nový prístup riadenia je založený na \uv{správaniach} --- jednoduchších, paralelne vykonávaných, interagujúcich moduloch, neustále vyhodnocujúcich situáciu (vstupy), ktoré transformujú na adekvátny výstup. Priamejšie prepojenie vstupu a výstupu na nižšej úrovni a upustenie od vytvárania zložitých modelov okolia prinieslo žiadanú pamäťovú i výpočtovú úsporu.

Zaujímavou sférou robotiky je takzvaná \uv{sociálna robotika}, zameraná na vyššiu úroveň interakcie prístrojov a človeka \cite{breazeal2016social}. Sociálny robot môže byť vnímaný viac ako spolupracovník, nie ako nástroj. Ich uplatnenie je najmä v oblasti vzdelávania či zdravotníctva. Za zakladateľa sociálnej robotiky je považovaný americký neurofyziológ William Grey Walter \cite{WalterTortoise}. Výsledkom jeho snažení bol mimo iného vznik jedných z prvých autonómnych robotov \uv{Machina Speculatrix}, známych tiež ako \uv{korytnačky}, ktoré sa samostatne pohybovali, nasledujúc zdroj svetla.

Odvetvím, v ktorom možno využitie robotov a pokrokov v ich vývoji sledovať sa stal vesmírny výskum. Prvenstvo patrí Sovietskemu zväzu, ktorý vyslal prvú robotickú vesmírnu loď už v roku 1951. Súboj o ďalšie úspechy, najmä so Spojenými štátmi, upriamil na robotiku pozornosť. Roboty zmapovali najbližšie planéty, ich životnosť na povrchu cieľových planét sa predlžovala z desiatok minút na roky. Dnes sa prvenstvom --- prvým humanoidným robotom vo vesmíre pýši americká agentúra NASA (angl. National Aeronautics and Space Administration --- Národný úrad pre letectvo a vesmír).


% Súčasnosť
\section{Súčasnosť}
Roboty sú dnes integrované v mnohých oblastiach ľudskej činnosti. Od jednoduchších systémov až po komplexnejšie, či tie ovládané umelou inteligenciou, roboty a s nimi spätá automatizácia nám pomáhajú šetriť náklady, zvyšujú produktivitu, eliminujú chyby. Pokročila i vízia robotov v umeleckej tvorbe, ktorá veští robotom do budúcna zlatú éru. V praxi možno najväčší prínos a najrozsiahlejšiu aplikáciu robotov vidieť v odvetviach priemyslu a medicíny. Roboty sú tiež vo veľkom využívané armádou, kde sa ako aj v iných oblastiach uplatňujú najmä pre presnosť a neúnavnosť. Rozšírila sa i teoretická časť robotiky a s pribúdajúcimi odvetviami aplikácie robotov vznikla aj potreba ich kategorizácie podľa možností mobility, účelu i technického vybavenia. Pomerne novou sférou sú lietajúce roboty, drony.

S nástupom robotiky vznikla tiež nutnosť výchovy expertov v tejto oblasti. Na tvorbe a prevádzke robota sa podieľajú odborníci z rôznych odborov, dôležitá je zložka dizajnu, konštrukcie i softvérového vybavenia obslužného programu. Učebné plány našich škôl sa však žiaľ nie vždy adaptujú na nové požiadavky v praxi s prijateľným oneskorením a v dostatočnej miere. Podpora robotiky je v úzadí najmä v prípade študentov základných škôl a gymnázií, vznikajú preto rôzne inštitúcie a projekty s cieľom motivovať potenciálnych záujemcov. Tu predstavujeme dva projekty, ktorých snažením je inšpirovaná i naša práca.

\subsubsection*{IT Akadémia}
IT Akadémia je jedným z národných projektov Centra vedecko--technických informácií SR (CVTI SR) \cite{CVTISR-ITAkademia}. CVTI SR je organizácia koordinujúca činnosť interdisciplinárnych výskumno-vývojových centier a národných infraštruktúr pre výskum, vývoj, inovácie a vzdelávanie \cite{CVTISR}. Mimo iného sa venuje aj popularizácii vedy a techniky, s čím súvisí projekt IT Akadémie. Ako názov napovedá, jedným zo zámerov je podpora vzdelávania v oblasti informačných a komunikačných technológií (IKT) \cite{ITAkademia}. Cieľovú skupinu tvoria žiaci základných a stredných škôl. Podpora spočíva v modernizácii obsahu a metód výučby predmetov súvisiacich s IKT, rozširovaní palety dostupných študijných materiálov, ale i v dopĺňaní vedomostí vyučujúcich. Mladých majú k ďalšiemu štúdiu oblasti motivovať i krúžky, semináre, súťaže či tábory.

\newpage
\subsubsection*{Fablab}
Fablab je celosvetová sieť laboratórií --- dielní, voľne dostupných verejnosti, poskytujúcich prístup k moderným technológiám ako 3D tlač, CNC frézovanie, alebo 3D skenovanie \cite{Fablab}. Cieľom je poskytnúť prístup k prostriedkom komukoľvek, kto má záujem o rozvoj alebo vzdelanie v oblasti. Priestory sú k dispozícii aj pre testovanie prototypov. Fablab Bratislava je organizačnou súčasťou CVTI SR, v prevádzke je od roku 2014 \cite{FablabVedeckyParkUK}. Dielňa podporuje rôzne projekty a v rámci činnosti tiež organizuje tvorivé dielne, sústredenia a tábory pre mladších účastníkov vo veku 11-15 rokov, kde je im umožnené zoznámiť sa s novou technológiou, programovaním i robotikou hravou formou \cite{FablabDTDT}. Pod názvom Denný tábor digitálnych technológií (DTDT) je jedna z takýchto akcií organizovaná každoročne od leta 2018.

