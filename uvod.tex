
\chapter*{Úvod} % chapter* je necislovana kapitola
\addcontentsline{toc}{chapter}{Úvod} % rucne pridanie do obsahu
\markboth{Úvod}{Úvod} % vyriesenie hlaviciek

Informatika je v dnešných časoch rozmáhajúcou sa vedou. Aplikáciou do praxe nám často uľahčuje prácu, šetrí čas. Pokrok v tejto oblasti je však (ako snáď aj v každej inej) podmienený neprestajným výskumom a objavovaním nových možností. Neoddeliteľnou súčasťou tohto procesu sú najmä ľudia investujúci doň čas, energiu a poznatky. Kde však takých nájsť? O svoju priazeň v očiach žiakov súperia dnes v edukačnom procese mnohé zaujímavé disciplíny.

Našim zámerom je podporiť výučbu robotiky a programovania u mladších, začínajúcich programátorov. Tieto odvetvia, veľmi príbuzné informatike, sú na edukačné účely priam ideálne --- vo vhodnej forme umožňujú pomerne jednoducho, hravou formou a interakciou vzbudiť záujem a motiváciu pre ďalšiu prácu \cite{parker2019learning}. Cieľom je vytvorenie bádateľského prostredia umožňujúceho žiakom objavovať svet v tvorivom a konštrukcionistickom vzdelávacom procese, platformy, ktorá podporí ich osobnostný rozvoj po rozličných stránkach --- z hľadiska priestorovej predstavivosti, abstraktného a formálneho myslenia, ale i získania praktických skúseností s artefaktmi z reálneho sveta. Používateľ tak získa cenné poznatky pre štúdium teoretických princípov všetkých zapojených oblastí, najmä fyziky, informatiky, matematiky ale i iných predmetov.

Východiskom sú dva existujúce roboty vytvorené na 3D tlačiarni, Otto a Mokrarosa, ktorých prednosťami sú predovšetkým cenová dostupnosť a modularita. Momentálna konštrukcia umožňuje pohyb končatín, čím je možné vykonávať chôdzu, otáčanie (zmenu smeru) a tvoriť rôzne choreografie. Roboty sú tiež osadené jednoduchým mp3 prehrávačom a ultrazvukovým senzorom, umožňujúcim merať vzdialenosť od prekážky. Riadenie prístroja zabezpečuje mikropočítač Arduino Nano Strong.

Úskalím pre nováčikov v oblasti je zadávanie príkazov a programovanie robota, čo možno toho času vykonávať len cez znakový terminál. Znaky sa odosielajú cez sériový port do riadiacej jednotky prístroja, kde sú dekódované a interpretované napríklad ako pohyb konkrétneho motora. Cieľom práce je túto prekážku odstrániť vytvorením desktopovej aplikácie s grafickým používateľským rozhraním, ktoré by nahradilo rolu terminálu v procese interakcie s robotom. Hlavným cieľom je umožniť programovanie v prehľadnom grafickom jazyku. Tento typ jazyka pripomína puzzle, kde jednotlivými dielikmi sú komponenty bežného programovacieho jazyka (ako napríklad podmienka if alebo cykly či konštanty), ktoré do seba zapadajú, a tak je možné tvoriť kód. Viac o ňom uvádzame v kapitole \ref{kap:GrafickyProgramovaciJayzk}.
Zámerom je tiež vytvoriť v aplikácii modul vizualizácie robota, v ktorom by bolo možné spúšťať vytvorené programy v jednoduchej simulácii bez prítomnosti robota samotného a modul ponúkajúci ukážkové programy, uľahčujúce prvotné zoznámenie s možnosťami aplikácie a robota.
