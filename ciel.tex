\chapter{Cieľ}

\label{kap:ciel}

Tu je opísaný predmet riešenia, ciele práce a požiadavky na výsledný produkt.

\section{Primárny cieľ}
Ústredným cieľom tejto práce je vytvorenie desktopovej aplikácie s grafickým používateľským prostredím (angl. GUI --- graphical user interface), ktoré uľahčí programovanie robotov Otto a Mokrarosa, opísaných v časti \ref{sec:OttoMokrarosa}. Ako prioritu stanovujeme podporu pre robota Otto, ktorý je konštrukciou o čosi jednoduchší. Programovanie zariadení máme v úmysle zjednodušiť najmä použitím vizuálneho programovacieho jazyka (VPJ), o ktorom viac uvádzame v kapitole \ref{kap:GrafickyProgramovaciJayzk}.

Inšpiráciou je nám predovšetkým aplikácia vytvorená konštruktérmi pôvodnej verzie robota Otto (Otto Blockly, časť \ref{sub:OttoBlockly}), ktorá spomínaný typ jazyka využíva. V základnej verzii našej aplikácie je žiaduce vytvoriť prvky podporujúce funkcionality dostupné v aplikácii Otto Blockly. Jedná sa najmä o prvky obsiahnuté vo VPJ, umožňujúce jednoduchý pohyb motormi, obsluhu senzorov, ale i komplexnejšie funkcie. V ďalšom je tiež potrebné umožniť kompiláciu vytvoreného kódu a jeho nahratie do riadiacej jednotky robota. Nezabúdame ani na funkciu umožňujúcu jednoduchú sériovú komunikáciu s riadiacou jednotkou pripojením rozhrania USB a možnosť načítania predpripravených ukážkových programov.

Výraznejšie sa odlíšiť od existujúcej aplikácie máme v pláne tvorbou modulu vizualizácie, v ktorom by bolo možné spúšťať vytvorené programy v jednoduchej simulácii. Pod \uv{jednoduchosťou} si predstavujeme zobrazenie tela robota v trojrozmernom priestore, umožnenie pohybu jednotlivými časťami (končatinami) na základe vytvoreného programu, a pôsobenie gravitácie na zobrazovaný model. V simulácii je prirodzene snahou čo najvierohodnejšie napodobniť realitu.

K odlíšeniu od aplikácie Otto Blockly prispeje tiež podpora tvorby choreografií ako postupnosti pohybov uchovávanej čisto v operačnej pamäti bez nutnosti kompilácie.

\section{Požiadavky}
Okrem základných (implicitných) požiadaviek na spoľahlivosť a funkčnosť je dôležité zdôrazniť ďalšie, plynúce z určenia cieľovej skupiny používateľov novovzniknutej aplikácie. Používateľmi má byť najmä veková kategória 10-15 rokov, čo je pomerne široké rozpätie. Potvrdila nám to i doc. PaedDr. Monika Tomcsányiová, PhD. z katedry didaktiky matematiky, fyziky a informatiky. Problémom je, že interval zahŕňa rôzne kognitívne vývinové štádiá žiakov, ktoré sa výrazne líšia napríklad v schopnosti spracovať abstrakciu. Je teda potrebné, aby aplikácia poskytla vo VPJ prvky zohľadňujúce túto skutočnosť a mladším programátorom umožnila jednoduchšiu tvorbu kódu.